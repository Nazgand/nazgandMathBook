\documentclass[]{article}
%made from template named MathArticleTemplate
\usepackage{amsfonts}
\usepackage{amsmath}
\usepackage{amsthm}
\usepackage{amssymb}
\usepackage{hyperref}
\hypersetup{colorlinks=true}
\usepackage{graphics}

%Fix \eqref in section title
\pdfstringdefDisableCommands{\def\eqref#1{(\ref{#1})}}

\DeclareMathOperator{\es}{Es}
\DeclareMathOperator{\ez}{Ez}
\DeclareMathOperator{\gs}{gs}
\DeclareMathOperator{\md}{mod}
\DeclareMathOperator{\pow}{Pow}
%Parenthesis, Braces, Brackets usepackage{physics}
\newcommand{\pqty}[1]{{\left(#1\right)}}
\newcommand{\Bqty}[1]{{\left\{#1\right\}}}
\newcommand{\bqty}[1]{{\left[#1\right]}}
\newcommand{\abs}[1]{{\left\lvert#1\right\rvert}}
%other stuff
\newcommand{\floor}[1]{{\left\lfloor#1\right\rfloor}}
\newcommand{\ceil}[1]{{\left\lceil#1\right\rceil}}
%Laplace transform and inverse
\newcommand{\laplace}[1]{\mathcal{L}\Bqty{#1}\pqty{s}}
\newcommand{\laplaceInv}[1]{\mathcal{L}^{-1}\Bqty{#1}\pqty{t}}
%Derivatives
\newcommand{\pdiff}[2]{\frac{\partial^{#2}}{\partial #1^{#2}}}

%lemma,theorem, proof
\newtheorem{theorem}{Theorem}[section]
\newtheorem{lemma}[theorem]{Lemma}
\newtheorem{definition}[theorem]{Definition}
\newtheorem{corollary}[theorem]{Corollary}

\numberwithin{equation}{section}

%\usepackage{minted}
%opening
\author{Mark Andrew Gerads: nazgand@gmail.com}

\title{\(\int_{0}^{\infty}\sin\pqty{x*f\pqty{x}}\partial x\) converges for positive monotone increasing unbounded \(f\pqty{x}\)}

\begin{document}
	
	\maketitle
	
	\begin{definition}
		Let a speed function \(f:\mathbb{R}^+\to\mathbb{R}^+\) exist such that
		\begin{equation}
		\bqty{\Bqty{a,b}\subseteq\mathbb{R}^+\land a<b}
		\Rightarrow
		f\pqty{a}<f\pqty{b}
		\end{equation}
	\end{definition}

	As \(f\pqty{x}\to\infty\), the period of \(\sin\pqty{x*f\pqty{x}}\) approaches zero, making the integral converge. More rigor below.
	
	\begin{lemma}[Discrete increases for \(f\) are enough for proof.]
		\begin{equation}
		\int_{0}^{\infty}\sin\pqty{x*f\pqty{x}}\partial x
		=
		\lim\limits_{n\to\infty}\int_{0}^{\infty}\sin\pqty{x*g_n\pqty{x}}\partial x
		\end{equation}
		where
		\begin{equation}
			g_n\pqty{x}=\frac{\floor{n*f\pqty{x}}}{n}
		\end{equation}
	\end{lemma}
	\begin{proof}
		Trivial.
	\end{proof}

	\begin{lemma}[More periods can be added so only 1 discontinuity need be considered at a time.]
		\begin{equation}
			\int_{0}^{\infty}\sin\pqty{x*f\pqty{x}}\partial x
			=
			\int_{0}^{\infty}\sin\pqty{x*g\pqty{x}}\partial x
		\end{equation}
		where \(y\in\mathbb{R}^+\) and
		\begin{equation}
			g\pqty{x}=
			\begin{cases}
				f\pqty{x} & \text{if } x<y\\
				f\pqty{y} & \text{if } y\leq x<y+\frac{2\pi}{f\pqty{y}}\\
				f\pqty{x-\frac{2\pi}{f\pqty{y}}} & \text{if } \frac{2\pi}{f\pqty{y}}\leq x
			\end{cases}
		\end{equation}
	\end{lemma}
	\begin{proof}
		\(g\) modifies \(f\) by adding a sinusoidal period to \(\sin\pqty{x*g\pqty{x}}\) which integrates to \(0\).
	\end{proof}


	These 2 Lemmas allow consideration of a period with 1 increase of speed (\(f\pqty{x}\)) without loss of generality. Showing that 2 periods will not produce more area than 1 period allows, by induction, the entire integral to be bounded by the area 1 period can produce.
	Let \(\sin\pqty{a*f\pqty{a}}=0\) be the start of a period. Let \(b\) be the point in the period where the speed increases to \(f\pqty{b}\). \(a<b<\frac{2\pi}{f\pqty{a}}\). The end of the period is \(c=b + \pqty{\frac{2\pi}{f\pqty{a}} + a - b}*\frac{f\pqty{a}}{f\pqty{b}}\). Let a further half period be at \(d=c+\frac{\pi}{f\pqty{b}}\).
	\begin{equation}
	f\pqty{x}=
	\begin{cases}
	f\pqty{x} & \text{if } x \leq a\\
	f\pqty{a} & \text{if } a \leq x < b\\
	f\pqty{b} & \text{if } b \leq x < d\\
	f\pqty{x} & \text{if } d \leq x
	\end{cases}
	\end{equation}
	The area of the period is thus:
	\begin{equation}
	A=
	\int_{a}^{b}\sin\pqty{x*f\pqty{a}}\partial x+
	\int_{b}^{c}\sin\pqty{x*f\pqty{b}}\partial x
	\end{equation}
	
	\begin{theorem}
		The maximum area the described period can have is \(\frac{2}{f\pqty{a}}\).
	\end{theorem}
	\begin{proof}
		The way to maximize the area of 1 period is to have a constant speed until half a period passes and the integral becomes negative, then have infinite speed. \(b=a+\frac{\pi}{f\pqty{a}},c=b,f\pqty{b}=\infty\)
		\begin{equation}
		A \leq \int_{a}^{a+\frac{\pi}{f\pqty{a}}}\sin\pqty{x*f\pqty{a}}\partial x
		+\int_{a+\frac{\pi}{f\pqty{a}}}^{a+\frac{\pi}{f\pqty{a}}}\sin\pqty{x*f\pqty{b}}\partial x
		=\frac{2}{f\pqty{a}}
		\end{equation}
	\end{proof}
	\begin{theorem}
	The maximum area the described 2 periods can have is \(\frac{2}{f\pqty{a}}\).
	\end{theorem}
	\begin{proof}
	The way to maximize the area of 2 periods is to have a constant speed until half a period passes, then have speed \(f\pqty{b}\) and the integral becomes negative, then have infinite speed at \(d\). \(b=a+\frac{\pi}{f\pqty{a}},c=b,f\pqty{d}=\infty\).
	\begin{equation}
	A \leq \int_{a}^{a+\frac{\pi}{f\pqty{a}}}\sin\pqty{x*f\pqty{a}}\partial x
	+\int_{b}^{d}\sin\pqty{x*f\pqty{b}}\partial x
	=\frac{2}{f\pqty{a}}
	\end{equation}
	Note \(\int_{b}^{d}\sin\pqty{x*f\pqty{b}}\partial x=0\).
	\end{proof}
	
	We can now see that adding more periods simply delays the inevitable because \(\lim\limits_{x\to\infty}\frac{2}{f\pqty{x}}=0\). Thus \(\int_{0}^{\infty}\sin\pqty{x*f\pqty{x}}\partial x\) converges as was to be shown.

	Bounds on the integral, where \(y\) may be any position in a period:
	\begin{equation}
	\frac{2}{f\pqty{y}}
	\geq
	\int_{y}^{\infty}\sin\pqty{x*f\pqty{x}}\partial x
	\geq
	-\frac{2}{f\pqty{y}}
	\end{equation}
	The lower bound can come from a symmetric argument with a period of \(-\sin\pqty{x}\) instead of \(\sin\pqty{x}\).

\end{document}
